%%%%%%%%%%%%%%%%%%%%%%%%%%%%%%%%%%%%%%%%%%%%%%
%
%		My Thesis
%
%		EDOC Template
%		2011
%
%%%%%%%%%%%%%%%%%%%%%%%%%%%%%%%%%%%%%%%%%%%%%%


%%%%%%%%%%%%%%%%%%%%%%%%%%%%%%%%%%%%%%%%%%%%%%
%
%		Thesis Settings
%
%		EDOC Template
%		2011
%
%%%%%%%%%%%%%%%%%%%%%%%%%%%%%%%%%%%%%%%%%%%%%%
% !TEX root = ../my_thesis.tex
\documentclass[a4paper,11pt]{book}

\usepackage[T1]{fontenc}
\usepackage[utf8]{inputenc}
% % Uncomment for bibliography
% % Bibliography using Biblatex
%\usepackage{doi}
%\usepackage[autostyle]{csquotes}
% \usepackage[
%     backend=biber,
%     style=authoryear,
%     natbib=true,
%     firstinits=true,
%     sortlocale=en_US,
%     url=false, 
%     doi=true,
%     eprint=false,
%     isbn=false
% ]{biblatex}
%\addbibresource{tail/bibliography.bib}
% % OR Bibliography management for Bibtex 
% Load natbib before babel
\usepackage[round]{natbib}


\usepackage[french,german,english]{babel}


%%%%%%%%%%%%%%%%%%%%%%%%%%%%%%%%%%%%%%%%%%%%%%%
%% EDOC THESIS TEMPLATE: Variant 1.0 -> Latin modern, large text width&height
%%%%%%%%%%%%%%%%%%%%%%%%%%%%%%%%%%%%%%%%%%%%%%%
\usepackage{lmodern} % use this to fix blurry typewriter text font
%\usepackage[a4paper,top=22mm,bottom=28mm,inner=35mm,outer=25mm]{geometry}
%%%%%%%%%%%%%%%%%%%%%%%%%%%%%%%%%%%%%%%%%%%%%%%

%%%%%%%%%%%%%%%%%%%%%%%%%%%%%%%%%%%%%%%%%%%%%%
% EDOC THESIS TEMPLATE: Variant 2.0 -> Utopia, Gabarrit A (lighter pages)
%%%%%%%%%%%%%%%%%%%%%%%%%%%%%%%%%%%%%%%%%%%%%%
\usepackage{fourier} % Utopia font-typesetting including mathematical formula compatible with newer TeX-Distributions (>2010)
%\usepackage{utopia} % on older systems -> use this package instead of fourier in combination with mathdesign for better looking results
%\usepackage[adobe-utopia]{mathdesign}
\setlength{\textwidth}{146.8mm} % = 210mm - 37mm - 26.2mm
\setlength{\oddsidemargin}{11.6mm} % 37mm - 1in (from hoffset)
\setlength{\evensidemargin}{0.8mm} % = 26.2mm - 1in (from hoffset)
\setlength{\topmargin}{-2.2mm} % = 0mm -1in + 23.2mm 
\setlength{\textheight}{221.9mm} % = 297mm -29.5mm -31.6mm - 14mm (12 to accomodate footline with pagenumber)
\setlength{\headheight}{14mm}
%%%%%%%%%%%%%%%%%%%%%%%%%%%%%%%%%%%%%%%%%%%%%%


\usepackage{setspace} % increase interline spacing slightly
\setstretch{1.1}

\makeatletter
\setlength{\@fptop}{0pt}  % for aligning all floating figures/tables etc... to the top margin
\makeatother


\usepackage{graphicx,xcolor}
\graphicspath{{images/}}

\usepackage{subfig}
\usepackage{booktabs}
\usepackage{lipsum}
\usepackage{microtype}
\usepackage{url}

\usepackage{fancyhdr}
\renewcommand{\sectionmark}[1]{\markright{\thesection\ #1}}
\pagestyle{fancy}
	\fancyhf{}
	\renewcommand{\headrulewidth}{0.4pt}
	\renewcommand{\footrulewidth}{0pt}
	\fancyhead[OR]{\bfseries \nouppercase{\rightmark}}
	\fancyhead[EL]{\bfseries \nouppercase{\leftmark}}
	\fancyfoot[EL,OR]{\thepage}
\fancypagestyle{plain}{
	\fancyhf{}
	\renewcommand{\headrulewidth}{0pt}
	\renewcommand{\footrulewidth}{0pt}
	\fancyfoot[EL,OR]{\thepage}}
\fancypagestyle{addpagenumbersforpdfimports}{
	\fancyhead{}
	\renewcommand{\headrulewidth}{0pt}
	\fancyfoot{}
	\fancyfoot[RO,LE]{\thepage}
}

\usepackage{listings}
\lstset{language=[LaTeX]Tex,tabsize=4, basicstyle=\scriptsize\ttfamily, showstringspaces=false, numbers=left, numberstyle=\tiny, numbersep=10pt, breaklines=true, breakautoindent=true, breakindent=10pt}

\usepackage{hyperref}
\hypersetup{pdfborder={0 0 0},
	colorlinks=true,
	linkcolor=black,
	citecolor=black,
	urlcolor=black}
\urlstyle{same}
\ifpdf
\usepackage[final]{pdfpages}
\else
\usepackage{calc}
\usepackage{breakurl}
\usepackage[nlwarning=false]{hypdvips}
\usepackage{backref}
\renewcommand*{\backref}[1]{}
\fi
\usepackage{bookmark}

\makeatletter
\renewcommand\@pnumwidth{20pt}
\makeatother

\makeatletter
\def\cleardoublepage{\clearpage\if@twoside \ifodd\c@page\else
    \hbox{}
    \thispagestyle{empty}
    \newpage
    \if@twocolumn\hbox{}\newpage\fi\fi\fi}
\makeatother \clearpage{\pagestyle{plain}\cleardoublepage}


%%%%% CHAPTER HEADER %%%%
\usepackage{color}
\usepackage{tikz}
\usepackage[explicit]{titlesec}
\newcommand*\chapterlabel{}
%\renewcommand{\thechapter}{\Roman{chapter}}
\titleformat{\chapter}[display]  % type (section,chapter,etc...) to vary,  shape (eg display-type)
	{\normalfont\bfseries\Huge} % format of the chapter
	{\gdef\chapterlabel{\thechapter\ }}     % the label 
 	{0pt} % separation between label and chapter-title
 	  {\begin{tikzpicture}[remember picture,overlay]
    \node[yshift=-8cm] at (current page.north west)
      {\begin{tikzpicture}[remember picture, overlay]
        \draw[fill=black] (0,0) rectangle(35.5mm,15mm);
        \node[anchor=north east,yshift=-7.2cm,xshift=34mm,minimum height=30mm,inner sep=0mm] at (current page.north west)
        {\parbox[top][30mm][t]{15mm}{\raggedleft \rule{0cm}{0.6cm}\color{white}\chapterlabel}};  %the empty rule is just to get better base-line alignment
        \node[anchor=north west,yshift=-7.2cm,xshift=37mm,text width=\textwidth,minimum height=30mm,inner sep=0mm] at (current page.north west)
              {\parbox[top][30mm][t]{\textwidth}{\rule{0cm}{0.6cm}\color{black}#1}};
       \end{tikzpicture}
      };
   \end{tikzpicture}
   \gdef\chapterlabel{}
  } % code before the title body
\titlespacing*{name=\chapter,numberless}{-3.7cm}{83.2pt-\parskip}{-3.2pt+\parskip}
\titlespacing*{\chapter}{-3.7cm}{50pt-\parskip-\parskip}{30pt+\parskip+\parskip}
\titlespacing*{\section}{0pt}{13.2pt}{1em-\parskip}  % 13.2pt is line spacing for a text with 11pt font size
\titlespacing*{\subsection}{0pt}{13.2pt}{1em-\parskip}
\titlespacing*{\subsubsection}{0pt}{13.2pt}{1em-\parskip}
\titlespacing*{\paragraph}{0pt}{13.2pt}{1em-\parskip}

\newcounter{myparts}
\newcommand*\partlabel{}
\titleformat{\part}[display]  % type (section,chapter,etc...) to vary,  shape (eg display-type)
	{\normalfont\bfseries\Huge} % format of the part
	{\gdef\partlabel{\thepart\ }}     % the label 
 	{0pt} % separation between label and part-title
 	  {\ifpdf\setlength{\unitlength}{20mm}\else\setlength{\unitlength}{0mm}\fi
	  \addtocounter{myparts}{1}
	  \begin{tikzpicture}[remember picture,overlay]
    \node[anchor=north west,xshift=-65mm,yshift=-6.9cm-\value{myparts}*20mm] at (current page.north east) % for unknown reasons: 3mm missing -> 65 instead of 62
      {\begin{tikzpicture}[remember picture, overlay]
        \draw[fill=black] (0,0) rectangle(62mm,20mm);   % -\value{myparts}\unitlength
        \node[anchor=north west,yshift=-6.1cm-\value{myparts}*\unitlength,xshift=-60.5mm,minimum height=30mm,inner sep=0mm] at (current page.north east)
        {\parbox[top][30mm][t]{55mm}{\raggedright \color{white}Part \partlabel \rule{0cm}{0.6cm}}};  %the empty rule is just to get better base-line alignment
        \node[anchor=north east,yshift=-6.1cm-\value{myparts}*\unitlength,xshift=-63.5mm,text width=\textwidth,minimum height=30mm,inner sep=0mm] at (current page.north east)
              {\parbox[top][30mm][t]{\textwidth}{\raggedleft \rule{0cm}{0.6cm}\color{black}#1}};
       \end{tikzpicture}
      };
   \end{tikzpicture}
   \gdef\partlabel{}
  } % code before the title body
\titlespacing*{\part}{11.06cm}{26.4pt-\parskip-\parskip}{0pt}

\usepackage{amsmath}
\usepackage{amsfonts}
\usepackage{amssymb}
\usepackage{mathtools}
% Fix the problem with delimiter size caused by fourier and amsmath packages.
\makeatletter
\def\resetMathstrut@{%
  \setbox\z@\hbox{%
    \mathchardef\@tempa\mathcode`\(\relax
      \def\@tempb##1"##2##3{\the\textfont"##3\char"}%
      \expandafter\@tempb\meaning\@tempa \relax
  }%
  \ht\Mathstrutbox@1.2\ht\z@ \dp\Mathstrutbox@1.2\dp\z@
}
\makeatother
%%%%%%%%%%%%%%%%%%%%%%%%%%%%%%%%%%%%%%%%%%%%%%
%
%		Thesis Settings
%		Custom settings
%
%		2011
%
%%%%%%%%%%%%%%%%%%%%%%%%%%%%%%%%%%%%%%%%%%%%%%

% %
% %   Use this file for your own custom packages, command-definitions, etc...
% %
% % 
% % Packages for references - cleverref must be last
% \usepackage{nameref}
% \usepackage{hyperref}
% \usepackage{cleveref}
% \usepackage[shortlabels]{enumitem}
% % Reduce spacing in bibliography
% \setlength{\bibsep}{0pt plus 0.3ex}
% % Allow equations to break between pages
% \allowdisplaybreaks
% % Penalty for widow and orphan
% \widowpenalty=9999
% \clubpenalty=9999
% %Penalty for relation and binary operation breaks in equations
% \relpenalty=9999
% \binoppenalty=9999

\usepackage{bussproofs}
\usepackage{subcaption}
\usepackage{CJKutf8}
\usepackage{stmaryrd}
\usepackage{verbatim}
\usepackage{tikz}
\usepackage{tikz-qtree}
\usepackage{multirow}
% \usepackage{subfiles}
\setlength\heavyrulewidth{0.25ex}  % place your custom packages, etc... in this file!


%%%%%%%%%%%%%%%%%%%%%%%%%%%%%%%%%%%%%%%%%%%%%%
%%%%% HEAD: Book-Begin
%%%%%%%%%%%%%%%%%%%%%%%%%%%%%%%%%%%%%%%%%%%%%%
\begin{document}
\setlength{\parindent}{0pt}
\setlength{\parskip}{0pt} %(needs to be before titlepage and frontmatter to keep the table of contents lists short)
% \frontmatter
% \input{head/titlepage.tex}
% \include{head/dedication}
% \setcounter{page}{0}
% \include{head/acknowledgements}
% \include{head/preface}
% \include{head/abstracts}

\cleardoublepage
\pdfbookmark{\contentsname}{toc}
\tableofcontents

% % Following content is OPTIONAL (List of figures + tables)
% \cleardoublepage
% \phantomsection
% \addcontentsline{toc}{chapter}{List of Figures} % adds an entry to the table of contents
% \listoffigures
% 
% \cleardoublepage
% \phantomsection
% \addcontentsline{toc}{chapter}{List of Tables} % adds an entry to the table of contents
% \listoftables
% your list of symbols here, if needed.


% space before each new paragraph according to the template guidelines.
%(needs to be after titlepage and frontmatter to keep the table of contents lists short)
\setlength{\parskip}{1em}


%%%%%%%%%%%%%%%%%%%%%%%%%%%%%%%%%%%%%%%%%%%%%%
%%%%% MAIN: The chapters of the thesis
%%%%%%%%%%%%%%%%%%%%%%%%%%%%%%%%%%%%%%%%%%%%%%
\mainmatter
\include{main/ch1_introduction}
\cleardoublepage
\part{Theory and Background}
\chapter{Core Aspects of Tonal Compositions}
    \section{Structured Tonal Motion}
    \section{Polyphony}
    \section{Musical Time}
    \section{Repetition}

\chapter{Meta-languages for Abstraction and Structure}
    \section{Description, Instantiation, and Interpretation} 
    \section{Grammar and Syntax}
        \subsection{Grammar of Natural Languages}
        \subsection{Grammar-Based Music Theory}
    \section{Combinatory Categorial Grammar}
    \section{Constructive Logic and Type Theory}
     

\chapter{Constructive Logic and Type theory}
    \section{Curry-Howard Correspondence}
    \section{Composable and Verifiable Derivations}
    \section{Abstract Context-Free Grammar as Dependent Type}
    \section{Modular Constructions beyond context-free Languages}
    \section{Generic Programming}
    
\chapter{Outlook}
    \section{Towards a Wholistic Model of Tonal Music}
        \subsection{Modular Theories that Work Together}
        \subsection{Relating General Principles to Style Specific Constraints}
    \section{Thesis Outline}
%\include{main/ch2_figures_tables}
\cleardoublepage
\part{A Process Algebra For Metered Polyphony}
\chapter{Tonal Motion as Hierarchical Processes}
    \section{Difficulties in the Assignment of Dependency Relations Among Entities}
    \section{Applicability and Inspiration From Existing Process Calculi}
        \subsection{Algebra of Communicating Processes}
        \subsection{Communicating Sequential Processes}
        \subsection{Calculus of Communicating Systems}
    \section{Stationary and Transitory Processes}
    \section{Temporal Abstractions}
    \section{Spatial Abstractions}
    \section{Connection with Existing Theories}

\chapter{Timely Processes}
    \section{Difficulties in Combining Tonal Structure and Metrical Structure}
        \subsection{Coordinating the metrical structure with pitch structure}
            \subsubsection{Passing Note in a Triple Meter}
            \subsubsection{Accented Passing Note}
            \subsubsection{Suspensions}
            \subsubsection{Interruptions}
        \subsection{Temporal Proportions: Phrasal expansion and contraction}
        
    \section{Temporal relations of processes}
        \subsection{Perceived vs Interpreted Simultaneity}
        \subsection{A Non-Linear Topology of Musical Time}
        \subsection{Modeling Meter as Nested Proportions}
        \subsection{Modeling Syncopation as Attending the Non-Present}
    \section{Connection with Existing Theories}


\part{General Structural Repetition}
    \section{Various Types of Structural Repetition}
        \subsection{Exact Repetition}
        \subsection{Surface Variation}
        \subsection{Reinterpretation}
        \subsection{Transference}

    \section{Existing Formalism of Repetition}
        \subsection{Without Syntactical Constraints}
        \subsection{With Syntactical Constraints}

    \section{A Formal Model of Structural Repetition}
        \subsection{Typed Holes and Arrows: A Functional Interpretation of Templates}
        \subsection{Repetition structure as Higher-Order Templates}
        \subsection{Repetition-Syntax-Material Decomposition}
        \subsection{Cross-Domain Analogies of Repetition Structure}

%\include{main/ch3_math}
%\include{main/ch4_more_text}
\cleardoublepage
\part{Computational Experiments}
    \section{Encoding and Generating Instances of Galant Schemata}
    \section{Sampling Metered Polyphony with Repetition Constraint}
    \section{Compression by Exploiting Structural Repetition}
% use to end the last part if the thesis is composed of parts
\addtocontents{toc}{\vspace{\normalbaselineskip}}
\cleardoublepage
\bookmarksetup{startatroot}

%%%%%%%%%%%%%%%%%%%%%%%%%%%%%%%%%%%%%%%%%%%%%%
%%%%% TAIL: Bibliography, Appendix, CV
%%%%%%%%%%%%%%%%%%%%%%%%%%%%%%%%%%%%%%%%%%%%%%
% \include{tail/appendix}
% \backmatter
% \include{tail/biblio}
% % Add your glossary here
% % Add your index here
% % Photographic credits (list of pictures&images that have been used with names of the person holding the copyright for them)
% \include{tail/cv}

\end{document}
