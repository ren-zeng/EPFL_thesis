%%%%%%%%%%%%%%%%%%%%%%%%%%%%%%%%%%%%%%%%%%%%%%
%
%		My Thesis
%
%		EDOC Template
%		2011
%
%%%%%%%%%%%%%%%%%%%%%%%%%%%%%%%%%%%%%%%%%%%%%%


\input{head/settings_epfl_template.tex}
\input{head/settings_custom.tex}  % place your custom packages, etc... in this file!


%%%%%%%%%%%%%%%%%%%%%%%%%%%%%%%%%%%%%%%%%%%%%%
%%%%% HEAD: Book-Begin
%%%%%%%%%%%%%%%%%%%%%%%%%%%%%%%%%%%%%%%%%%%%%%
\begin{document}
\setlength{\parindent}{0pt}
\setlength{\parskip}{0pt} %(needs to be before titlepage and frontmatter to keep the table of contents lists short)
% \frontmatter
% \input{head/titlepage.tex}
% \include{head/dedication}
% \setcounter{page}{0}
% \include{head/acknowledgements}
% \include{head/preface}
% \include{head/abstracts}

\cleardoublepage
\pdfbookmark{\contentsname}{toc}
\tableofcontents

% % Following content is OPTIONAL (List of figures + tables)
% \cleardoublepage
% \phantomsection
% \addcontentsline{toc}{chapter}{List of Figures} % adds an entry to the table of contents
% \listoffigures
% 
% \cleardoublepage
% \phantomsection
% \addcontentsline{toc}{chapter}{List of Tables} % adds an entry to the table of contents
% \listoftables
% your list of symbols here, if needed.


% space before each new paragraph according to the template guidelines.
%(needs to be after titlepage and frontmatter to keep the table of contents lists short)
\setlength{\parskip}{1em}


%%%%%%%%%%%%%%%%%%%%%%%%%%%%%%%%%%%%%%%%%%%%%%
%%%%% MAIN: The chapters of the thesis
%%%%%%%%%%%%%%%%%%%%%%%%%%%%%%%%%%%%%%%%%%%%%%
\mainmatter
\include{main/ch1_introduction}
\cleardoublepage
\part{Theory and Background}
\chapter{Core Aspects of Tonal Compositions}
    \section{Structured Tonal Motion}
    \section{Polyphony}
    \section{Musical Time}
    \section{Repetition}

\chapter{Meta-languages for Abstraction and Structure}
    \section{Description, Instantiation, and Interpretation} 
    \section{Grammar and Syntax}
        \subsection{Grammar of Natural Languages}
        \subsection{Grammar-Based Music Theory}
    \section{Combinatory Categorial Grammar}
    \section{Constructive Logic and Type Theory}
     

\chapter{Constructive Logic and Type theory}
    \section{Curry-Howard Correspondence}
    \section{Composable and Verifiable Derivations}
    \section{Abstract Context-Free Grammar as Dependent Type}
    \section{Modular Constructions beyond context-free Languages}
    \section{Generic Programming}
    
\chapter{Outlook}
    \section{Towards a Wholistic Model of Tonal Music}
        \subsection{Modular Theories that Work Together}
        \subsection{Relating General Principles to Style Specific Constraints}
    \section{Thesis Outline}
%\include{main/ch2_figures_tables}
\cleardoublepage
\part{A Process Algebra For Metered Polyphony}
\chapter{Tonal Motion as Hierarchical Processes}
    \section{Difficulties in the Assignment of Dependency Relations Among Entities}
    \section{Applicability and Inspiration From Existing Process Calculi}
        \subsection{Algebra of Communicating Processes}
        \subsection{Communicating Sequential Processes}
        \subsection{Calculus of Communicating Systems}
    \section{Stationary and Transitory Processes}
    \section{Temporal Abstractions}
    \section{Spatial Abstractions}
    \section{Connection with Existing Theories}

\chapter{Timely Processes}
    \section{Difficulties in Combining Tonal Structure and Metrical Structure}
        \subsection{Coordinating the metrical structure with pitch structure}
            \subsubsection{Passing Note in a Triple Meter}
            \subsubsection{Accented Passing Note}
            \subsubsection{Suspensions}
            \subsubsection{Interruptions}
        \subsection{Temporal Proportions: Phrasal expansion and contraction}
        
    \section{Temporal relations of processes}
        \subsection{Perceived vs Interpreted Simultaneity}
        \subsection{A Non-Linear Topology of Musical Time}
        \subsection{Modeling Meter as Nested Proportions}
        \subsection{Modeling Syncopation as Attending the Non-Present}
    \section{Connection with Existing Theories}


\part{General Structural Repetition}
    \section{Various Types of Structural Repetition}
        \subsection{Exact Repetition}
        \subsection{Surface Variation}
        \subsection{Reinterpretation}
        \subsection{Transference}

    \section{Existing Formalism of Repetition}
        \subsection{Without Syntactical Constraints}
        \subsection{With Syntactical Constraints}

    \section{A Formal Model of Structural Repetition}
        \subsection{Typed Holes and Arrows: A Functional Interpretation of Templates}
        \subsection{Repetition structure as Higher-Order Templates}
        \subsection{Repetition-Syntax-Material Decomposition}
        \subsection{Cross-Domain Analogies of Repetition Structure}

%\include{main/ch3_math}
%\include{main/ch4_more_text}
\cleardoublepage
\part{Computational Experiments}
    \section{Encoding and Generating Instances of Galant Schemata}
    \section{Sampling Metered Polyphony with Repetition Constraint}
    \section{Compression by Exploiting Structural Repetition}
% use to end the last part if the thesis is composed of parts
\addtocontents{toc}{\vspace{\normalbaselineskip}}
\cleardoublepage
\bookmarksetup{startatroot}

%%%%%%%%%%%%%%%%%%%%%%%%%%%%%%%%%%%%%%%%%%%%%%
%%%%% TAIL: Bibliography, Appendix, CV
%%%%%%%%%%%%%%%%%%%%%%%%%%%%%%%%%%%%%%%%%%%%%%
% \include{tail/appendix}
% \backmatter
% \include{tail/biblio}
% % Add your glossary here
% % Add your index here
% % Photographic credits (list of pictures&images that have been used with names of the person holding the copyright for them)
% \include{tail/cv}

\end{document}
